% OSPP 申请书示例代码库
% 包含各种常用的 LaTeX 示例和模板

% ========== 列表示例 ==========
% 基础列表
\begin{itemize}
    \item 第一项
    \item 第二项
    \item 第三项
\end{itemize}

% 编号列表
\begin{enumerate}
    \item 第一步
    \item 第二步
    \item 第三步
\end{enumerate}

% 自定义列表
\begin{itemize}[label=\textbullet]
    \item 使用自定义符号的列表项
    \item 另一个列表项
\end{itemize}

% ========== 表格示例 ==========
% 基础表格
\begin{table}[H]
\centering
\begin{tabular}{|c|c|c|}
\hline
\textbf{列1} & \textbf{列2} & \textbf{列3} \\
\hline
内容1 & 内容2 & 内容3 \\
\hline
内容4 & 内容5 & 内容6 \\
\hline
\end{tabular}
\caption{表格标题}
\label{tab:example}
\end{table}

% 复杂表格(适用于技术方案)
\begin{table}[H]
\centering
\begin{tabular}{|p{3cm}|p{4cm}|p{6cm}|}
\hline
\textbf{技术栈} & \textbf{版本要求} & \textbf{使用说明} \\
\hline
Python & 3.8+ & 主要开发语言,用于数据处理和算法实现 \\
\hline
React & 18.0+ & 前端框架,负责用户界面开发 \\
\hline
Docker & 20.10+ & 容器化部署,确保环境一致性 \\
\hline
\end{tabular}
\caption{技术栈详细说明}
\end{table}

% ========== 代码示例 ==========
% 行内代码
使用 \texttt{git clone} 命令克隆仓库。

% 代码块
\begin{verbatim}
def hello_world():
    print("Hello, OSPP!")
    return True
\end{verbatim}

% ========== 数学公式示例 ==========
% 行内公式
算法的时间复杂度为 $O(n \log n)$。

% 独立公式
\begin{equation}
E = mc^2
\end{equation}

% ========== 图片示例 ==========
% 插入图片(需要graphicx包)
\begin{figure}[H]
\centering
% \includegraphics[width=0.8\textwidth]{images/architecture.png}
\caption{系统架构图}
\label{fig:architecture}
\end{figure}

% ========== 引用示例 ==========
% 文献引用格式
根据研究\cite{author2023},该方法可以提升30\%的性能。

% 网址引用
更多信息请参考 \url{https://summer-ospp.ac.cn/}。

% ========== 颜色文本示例 ==========
\textcolor{blue}{蓝色文本}
\textcolor{red}{红色文本}
\textcolor{osppblue}{OSPP主题色文本}

% ========== 强调文本示例 ==========
\textbf{粗体文本}
\textit{斜体文本}
\underline{下划线文本}

% ========== 项目时间线示例 ==========
\begin{itemize}
    \item \textbf{第1周}:环境搭建与需求分析
    \begin{itemize}
        \item 安装开发环境
        \item 阅读项目文档
        \item 与导师初次沟通
    \end{itemize}
    \item \textbf{第2周}:详细设计
    \begin{itemize}
        \item 完成系统设计文档
        \item 确定技术选型
        \item 制定编码规范
    \end{itemize}
\end{itemize}

% ========== 技能评估表示例 ==========
\begin{table}[H]
\centering
\begin{tabular}{|l|c|c|c|c|}
\hline
\textbf{技能} & \textbf{初级} & \textbf{中级} & \textbf{高级} & \textbf{专家} \\
\hline
Python & & & \checkmark & \\
\hline
JavaScript & & \checkmark & & \\
\hline
Docker & & \checkmark & & \\
\hline
Git & & & \checkmark & \\
\hline
\end{tabular}
\caption{个人技能评估}
\end{table}

% ========== 风险评估矩阵示例 ==========
\begin{table}[H]
\centering
\begin{tabular}{|p{4cm}|c|c|p{5cm}|}
\hline
\textbf{风险} & \textbf{概率} & \textbf{影响} & \textbf{应对措施} \\
\hline
技术难度超出预期 & 中 & 高 & 提前学习相关技术,寻求导师指导 \\
\hline
第三方依赖问题 & 低 & 中 & 准备备选方案,定期更新依赖 \\
\hline
时间安排冲突 & 中 & 中 & 制定详细计划,预留缓冲时间 \\
\hline
\end{tabular}
\caption{项目风险评估矩阵}
\end{table} 