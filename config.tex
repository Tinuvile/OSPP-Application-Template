% OSPP 申请书配置文件
% 请在此文件中填写您的个人信息和项目信息

% ========== 个人基本信息 ==========
\renewcommand{\studentname}{张三}                    % 您的姓名
\renewcommand{\studentid}{2021123456}               % 学号
\renewcommand{\studentemail}{zhangsan@example.com}  % 邮箱
\renewcommand{\studentphone}{138-0000-0000}         % 电话
\renewcommand{\studentuniversity}{某某大学}          % 大学
\renewcommand{\studentmajor}{计算机科学与技术}       % 专业

% ========== 项目相关信息 ==========
\renewcommand{\projecttitle}{基于深度学习的开源项目漏洞检测系统}  % 项目标题
\renewcommand{\communityname}{Apache Software Foundation}        % 社区名称
\renewcommand{\mentorname}{李导师}                                % 导师姓名
\renewcommand{\mentoremail}{liteacher@apache.org}               % 导师邮箱
\renewcommand{\projectdifficulty}{进阶}                          % 项目难度:基础/进阶/高级
\renewcommand{\projectreward}{12000元}                          % 项目奖金

% ========== 页面格式设置 ==========
% 修复页眉高度警告
\setlength{\headheight}{14.5pt}

% 页面边距(如需调整请取消注释)
% \geometry{left=3cm,right=3cm,top=3cm,bottom=3cm}

% ========== 颜色设置 ==========
% OSPP 官方主题色(参考官网 https://summer-ospp.ac.cn/)
\definecolor{osppblue}{RGB}{51, 122, 183}      % OSPP主蓝色
\definecolor{osppgray}{RGB}{108, 117, 125}     % OSPP灰色
\definecolor{ospplight}{RGB}{248, 249, 250}    % OSPP浅色背景
\definecolor{osppdark}{RGB}{52, 58, 64}        % OSPP深色文字

% 辅助颜色
\definecolor{successgreen}{RGB}{40, 167, 69}   % 成功绿色
\definecolor{warningorange}{RGB}{255, 193, 7}  % 警告橙色
\definecolor{dangerred}{RGB}{220, 53, 69}      % 危险红色

% ========== 超链接颜色设置 ==========
\hypersetup{
    colorlinks=true,
    linkcolor=black,        % 目录链接改为黑色
    filecolor=osppblue,     
    urlcolor=osppblue,      % 网址链接使用OSPP蓝色
    citecolor=osppblue,     % 引用链接使用OSPP蓝色
    pdftitle={OSPP 2025 项目申请书 - \projecttitle},
    pdfauthor={\studentname},
    pdfsubject={开源软件供应链点亮计划 2025},
    pdfkeywords={OSPP, 开源, 申请书, \studentname}
}

% ========== 字体设置 ==========
% 如果需要使用特定字体,可以在此设置
% \setCJKmainfont{SimSun}                       % 设置中文主字体
% \setCJKsansfont{SimHei}                       % 设置中文无衬线字体

% ========== 自定义命令 ==========
% 定义一些常用的格式命令
\newcommand{\highlight}[1]{\textcolor{osppblue}{\textbf{#1}}}
\newcommand{\important}[1]{\textcolor{dangerred}{\textbf{#1}}}
\newcommand{\note}[1]{\textcolor{osppgray}{\textit{#1}}}
\newcommand{\code}[1]{\texttt{#1}}

% 定义项目相关的命令
\newcommand{\githuburl}{https://github.com/yourusername}
\newcommand{\giteeurl}{https://gitee.com/yourusername}

% ========== 表格样式设置 ==========
% 设置表格的默认样式
\renewcommand{\arraystretch}{1.2}  % 增加表格行间距

% ========== 列表样式设置 ==========
% 自定义列表项符号
\renewcommand{\labelitemi}{\textcolor{osppblue}{\textbullet}}
\renewcommand{\labelitemii}{\textcolor{osppgray}{\textendash}}

% ========== 段落设置 ==========
\setlength{\parindent}{2em}        % 段落缩进
\setlength{\parskip}{0.5em}        % 段落间距 