\documentclass[a4paper,12pt]{article}

% ========== 包引入 ==========
% 支持中文
\usepackage[UTF8]{ctex}
\usepackage{geometry}
\usepackage{graphicx}
\usepackage{hyperref}
\usepackage{fancyhdr}
\usepackage{titlesec}
\usepackage{enumitem}
\usepackage{tabularx}
\usepackage{colortbl}
\usepackage{xcolor}
\usepackage{amsmath}
\usepackage{amssymb}
\usepackage{float}
\usepackage{array}
\usepackage{booktabs}
\usepackage{multirow}
\usepackage{fontawesome5}  % 图标支持
\usepackage{tcolorbox}     % 彩色文本框

% ========== 页面设置 ==========
\geometry{left=2.5cm,right=2.5cm,top=2.5cm,bottom=2.5cm}

% ========== 自定义命令用于填写个人信息 ==========
\newcommand{\studentname}{您的姓名}
\newcommand{\studentid}{您的学号}
\newcommand{\studentemail}{your.email@example.com}
\newcommand{\studentphone}{您的电话}
\newcommand{\studentuniversity}{您的大学}
\newcommand{\studentmajor}{您的专业}

% ========== 项目相关信息(可选,用于简历中引用) ==========
\newcommand{\projecttitle}{基于深度学习的开源项目漏洞检测工具}
\newcommand{\communityname}{Apache Software Foundation}
\newcommand{\mentorname}{李导师}
\newcommand{\mentoremail}{liteacher@apache.org}
\newcommand{\projectdifficulty}{进阶}
\newcommand{\projectreward}{12000元}

% ========== 引入配置文件 ==========
\IfFileExists{config.tex}{% OSPP 申请书配置文件
% 请在此文件中填写您的个人信息和项目信息

% ========== 个人基本信息 ==========
\renewcommand{\studentname}{张三}                    % 您的姓名
\renewcommand{\studentid}{2021123456}               % 学号
\renewcommand{\studentemail}{zhangsan@example.com}  % 邮箱
\renewcommand{\studentphone}{138-0000-0000}         % 电话
\renewcommand{\studentuniversity}{某某大学}          % 大学
\renewcommand{\studentmajor}{计算机科学与技术}       % 专业

% ========== 项目相关信息 ==========
\renewcommand{\projecttitle}{基于深度学习的开源项目漏洞检测系统}  % 项目标题
\renewcommand{\communityname}{Apache Software Foundation}        % 社区名称
\renewcommand{\mentorname}{李导师}                                % 导师姓名
\renewcommand{\mentoremail}{liteacher@apache.org}               % 导师邮箱
\renewcommand{\projectdifficulty}{进阶}                          % 项目难度:基础/进阶/高级
\renewcommand{\projectreward}{12000元}                          % 项目奖金

% ========== 页面格式设置 ==========
% 修复页眉高度警告
\setlength{\headheight}{14.5pt}

% 页面边距(如需调整请取消注释)
% \geometry{left=3cm,right=3cm,top=3cm,bottom=3cm}

% ========== 颜色设置 ==========
% OSPP 官方主题色(参考官网 https://summer-ospp.ac.cn/)
\definecolor{osppblue}{RGB}{51, 122, 183}      % OSPP主蓝色
\definecolor{osppgray}{RGB}{108, 117, 125}     % OSPP灰色
\definecolor{ospplight}{RGB}{248, 249, 250}    % OSPP浅色背景
\definecolor{osppdark}{RGB}{52, 58, 64}        % OSPP深色文字

% 辅助颜色
\definecolor{successgreen}{RGB}{40, 167, 69}   % 成功绿色
\definecolor{warningorange}{RGB}{255, 193, 7}  % 警告橙色
\definecolor{dangerred}{RGB}{220, 53, 69}      % 危险红色

% ========== 超链接颜色设置 ==========
\hypersetup{
    colorlinks=true,
    linkcolor=black,        % 目录链接改为黑色
    filecolor=osppblue,     
    urlcolor=osppblue,      % 网址链接使用OSPP蓝色
    citecolor=osppblue,     % 引用链接使用OSPP蓝色
    pdftitle={OSPP 2025 项目申请书 - \projecttitle},
    pdfauthor={\studentname},
    pdfsubject={开源软件供应链点亮计划 2025},
    pdfkeywords={OSPP, 开源, 申请书, \studentname}
}

% ========== 字体设置 ==========
% 如果需要使用特定字体,可以在此设置
% \setCJKmainfont{SimSun}                       % 设置中文主字体
% \setCJKsansfont{SimHei}                       % 设置中文无衬线字体

% ========== 自定义命令 ==========
% 定义一些常用的格式命令
\newcommand{\highlight}[1]{\textcolor{osppblue}{\textbf{#1}}}
\newcommand{\important}[1]{\textcolor{dangerred}{\textbf{#1}}}
\newcommand{\note}[1]{\textcolor{osppgray}{\textit{#1}}}
\newcommand{\code}[1]{\texttt{#1}}

% 定义项目相关的命令
\newcommand{\githuburl}{https://github.com/yourusername}
\newcommand{\giteeurl}{https://gitee.com/yourusername}

% ========== 表格样式设置 ==========
% 设置表格的默认样式
\renewcommand{\arraystretch}{1.2}  % 增加表格行间距

% ========== 列表样式设置 ==========
% 自定义列表项符号
\renewcommand{\labelitemi}{\textcolor{osppblue}{\textbullet}}
\renewcommand{\labelitemii}{\textcolor{osppgray}{\textendash}}

% ========== 段落设置 ==========
\setlength{\parindent}{2em}        % 段落缩进
\setlength{\parskip}{0.5em}        % 段落间距 }{}

% ========== 基础超链接设置 ==========
\hypersetup{
    colorlinks=true,
    linkcolor=black,        % 目录链接设为黑色
    filecolor=magenta,      
    urlcolor=osppblue,      % 网址链接使用OSPP蓝色
    pdftitle={OSPP 2025 个人简历 - \studentname},
    pdfauthor={\studentname},
}

% ========== 页眉页脚设置 ==========
\pagestyle{fancy}
\fancyhf{}
\rhead{\small OSPP 2025 个人简历}
\lhead{\small \studentname}
\cfoot{\thepage}
\renewcommand{\headrulewidth}{0.4pt}

% ========== 标题格式设置 ==========
\titleformat{\section}{\Large\bfseries\color{osppblue}}{\thesection}{1em}{}
\titleformat{\subsection}{\large\bfseries\color{osppdark}}{\thesubsection}{1em}{}
\titleformat{\subsubsection}{\normalsize\bfseries\color{osppgray}}{\thesubsubsection}{1em}{}

\begin{document}

% ========== 标题页 ==========
\begin{titlepage}
    \centering
    
    % OSPP Logo 和官方标识
    \vspace{0.5cm}
    
    % OSPP图标(如果存在)
    \IfFileExists{image/OSPP.png}{
        \includegraphics[width=2.5cm]{image/OSPP.png} \\
        \vspace{0.5cm}
    }{}
    
    % 简历标题
    {\Huge \textcolor{osppblue}{\textbf{个人简历}}} \\
    \vspace{0.3cm}
    {\LARGE \textcolor{osppblue}{\textbf{Personal Resume}}} \\
    \vspace{0.2cm}
    {\Large \textcolor{osppgray}{\textbf{OSPP 2025 申请}}} \\
    
    \vspace{2cm}
    
    % 申请人信息
    {\huge \textcolor{osppdark}{\textbf{\studentname}}}
    
    \vspace{1cm}
    
    % 基本信息表格
    \begin{tabular}{rl}
        \textbf{\textcolor{osppblue}{学号:}} & \studentid \\[0.2cm]
        \textbf{邮箱:} & \studentemail \\[0.1cm]
        \textbf{电话:} & \studentphone \\[0.1cm]
        \textbf{大学:} & \studentuniversity \\[0.1cm]
        \textbf{专业:} & \studentmajor \\[0.2cm]
        \textbf{GitHub:} & \url{https://github.com/yourusername} \\[0.1cm]
        \textbf{Gitee:} & \url{https://gitee.com/yourusername} \\
    \end{tabular}
    
    \vfill
    
    {\large \textcolor{osppgray}{更新日期:\today}}
    
\end{titlepage}

% ========== 目录 ==========
\tableofcontents
\newpage

\section{基本信息}

\begin{table}[H]
\centering
\begin{tabular}{rl}
    \textbf{姓名:} & \studentname \\[0.2cm]
    \textbf{性别:} & 男/女 \\[0.1cm]
    \textbf{年龄:} & XX岁 \\[0.1cm]
    \textbf{邮箱:} & \studentemail \\[0.1cm]
    \textbf{电话:} & \studentphone \\[0.1cm]
    \textbf{学校:} & \studentuniversity \\[0.1cm]
    \textbf{专业:} & \studentmajor \\[0.1cm]
    \textbf{GitHub:} & \url{https://github.com/yourusername} \\[0.1cm]
    \textbf{Gitee:} & \url{https://gitee.com/yourusername} \\
\end{tabular}
\end{table}

\section{教育背景}

\subsection{学历信息}
\begin{itemize}
    \item \textbf{2021年9月 - 2025年6月:}\studentuniversity,\studentmajor,本科
    \item \textbf{主要课程:}数据结构与算法、软件工程、数据库系统、计算机网络、操作系统、编译原理、人工智能基础等
    \item \textbf{GPA:}X.XX/4.0(排名:XX/XXX)
    \item \textbf{英语水平:}CET-6(分数:XXX)/ 雅思X.X分 / 托福XXX分
\end{itemize}

\subsection{学术成果}
\begin{itemize}
    \item 参与导师科研项目:《XXXX研究》,负责算法设计与实现
    \item 发表学术论文:《XXXX》,第一作者,发表于XX会议/期刊
    \item 获得学术奖项:XX奖学金、优秀学生、三好学生等
\end{itemize}

\section{技术技能}

\subsection{编程语言}

\begin{table}[H]
\centering
\begin{tabular}{|l|c|c|c|c|l|}
\hline
\textbf{语言} & \textbf{熟练} & \textbf{中等} & \textbf{了解} & \textbf{使用年限} & \textbf{主要项目经验} \\
\hline
Python & \checkmark & & & 3年 & Web开发、数据分析 \\
\hline
Java & & \checkmark & & 2年 & 后端开发、Android应用 \\
\hline
JavaScript & & \checkmark & & 2年 & 前端开发、Node.js后端 \\
\hline
C++ & & & \checkmark & 1年 & 算法竞赛、系统编程 \\
\hline
Go & & & \checkmark & 0.5年 & 微服务开发、API设计 \\
\hline
\end{tabular}
\caption{编程语言技能评估}
\end{table}

\subsection{开发框架与工具}

\subsubsection{前端技术}
\begin{itemize}
    \item \textbf{框架:}React, Vue.js, Angular(基础)
    \item \textbf{样式:}Bootstrap, Tailwind CSS, Ant Design
    \item \textbf{构建工具:}Webpack, Vite, Create React App
    \item \textbf{状态管理:}Redux, Vuex, MobX
\end{itemize}

\subsubsection{后端技术}
\begin{itemize}
    \item \textbf{Python:}Django, Flask, FastAPI
    \item \textbf{Java:}Spring Boot, Spring MVC, MyBatis
    \item \textbf{Node.js:}Express, Koa, NestJS
    \item \textbf{API设计:}RESTful API, GraphQL, gRPC
\end{itemize}

\subsubsection{数据库技术}
\begin{itemize}
    \item \textbf{关系型:}MySQL, PostgreSQL, SQLite
    \item \textbf{非关系型:}MongoDB, Redis, Elasticsearch
    \item \textbf{数据分析:}Pandas, NumPy, Matplotlib, Jupyter
\end{itemize}

\subsubsection{开发工具}
\begin{itemize}
    \item \textbf{版本控制:}Git, GitHub, GitLab, SVN
    \item \textbf{IDE/编辑器:}VS Code, IntelliJ IDEA, PyCharm, Vim
    \item \textbf{容器化:}Docker, Docker Compose, Kubernetes(基础)
    \item \textbf{CI/CD:}GitHub Actions, Jenkins, GitLab CI
    \item \textbf{云平台:}AWS(基础), 腾讯云, 阿里云
\end{itemize}

\section{项目经验}

\subsection{开源贡献}
% 必填项:开源平台的用户ID(GitHub、Gitee等)
% 应该尽可能和自己所申请的项目相关

\subsubsection{项目1:为Apache OpenWhisk贡献代码}
\begin{itemize}
    \item \textbf{时间:}2024年6月 - 2024年10月
    \item \textbf{角色:}Active Contributor
    \item \textbf{贡献内容:}
    \begin{itemize}
        \item 修复了CLI工具中的3个关键bug,提升了用户体验
        \item 添加了新的认证模块,支持OAuth 2.0认证
        \item 优化了容器镜像构建流程,减少镜像大小30\%
        \item 编写了详细的API文档和使用教程
    \end{itemize}
    \item \textbf{技术栈:}Go, Docker, Kubernetes, OpenAPI
    \item \textbf{成果:}提交了12个Pull Request,8个被合并,获得社区好评
    \item \textbf{项目链接:}\url{https://github.com/apache/openwhisk}
    \item \textbf{个人贡献:}\url{https://github.com/apache/openwhisk/pulls?q=author:yourusername}
\end{itemize}

\subsubsection{项目2:参与React生态项目开发}
\begin{itemize}
    \item \textbf{时间:}2024年3月 - 2024年8月
    \item \textbf{角色:}Community Member
    \item \textbf{贡献内容:}
    \begin{itemize}
        \item 为React Router添加了TypeScript类型定义
        \item 翻译了官方文档的中文版本
        \item 参与Issue讨论,帮助新手解决问题
        \item 编写了多个使用示例和最佳实践
    \end{itemize}
    \item \textbf{技术栈:}React, TypeScript, JavaScript
    \item \textbf{社区活动:}参与了3次线上会议,分享技术经验
\end{itemize}

\subsection{个人项目}

\subsubsection{项目1:智能学习助手系统}
\begin{itemize}
    \item \textbf{时间:}2024年1月 - 2024年5月
    \item \textbf{项目描述:}基于机器学习的个性化学习推荐系统
    \item \textbf{主要功能:}
    \begin{itemize}
        \item 学习内容个性化推荐
        \item 学习进度跟踪和分析
        \item 智能错题本和复习提醒
        \item 学习效果可视化分析
    \end{itemize}
    \item \textbf{技术栈:}
    \begin{itemize}
        \item \textbf{后端:}Python, Django, PostgreSQL, Redis
        \item \textbf{前端:}React, TypeScript, Ant Design
        \item \textbf{机器学习:}scikit-learn, TensorFlow, pandas
        \item \textbf{部署:}Docker, Nginx, AWS EC2
    \end{itemize}
    \item \textbf{项目亮点:}
    \begin{itemize}
        \item 实现了协同过滤推荐算法,准确率达到85\%
        \item 支持10万+用户并发访问
        \item 获得校内创新大赛一等奖
    \end{itemize}
    \item \textbf{代码链接:}\url{https://github.com/yourusername/smart-learning-assistant}
    \item \textbf{演示地址:}\url{https://demo.smartlearning.com}
\end{itemize}

\subsubsection{项目2:分布式任务调度系统}
\begin{itemize}
    \item \textbf{时间:}2023年9月 - 2023年12月
    \item \textbf{项目描述:}高性能的分布式任务调度和执行系统
    \item \textbf{主要功能:}
    \begin{itemize}
        \item 任务的分布式调度和负载均衡
        \item 任务执行状态监控和日志收集
        \item 支持多种任务类型(定时、延时、循环)
        \item 故障恢复和数据一致性保证
    \end{itemize}
    \item \textbf{技术栈:}Go, etcd, gRPC, Prometheus, Docker
    \item \textbf{性能指标:}
    \begin{itemize}
        \item 支持每秒处理1000+任务
        \item 任务调度延迟 < 100ms
        \item 系统可用性 > 99.9\%
    \end{itemize}
    \item \textbf{代码链接:}\url{https://github.com/yourusername/distributed-scheduler}
\end{itemize}

\subsubsection{项目3:区块链投票系统}
\begin{itemize}
    \item \textbf{时间:}2023年6月 - 2023年8月
    \item \textbf{项目描述:}基于以太坊的去中心化投票DApp
    \item \textbf{技术栈:}Solidity, Web3.js, React, MetaMask, IPFS
    \item \textbf{特点:}投票结果不可篡改,完全透明公开
    \item \textbf{代码链接:}\url{https://github.com/yourusername/blockchain-voting}
\end{itemize}

\section{竞赛与获奖}

\subsection{编程竞赛}
\begin{itemize}
    \item \textbf{2024年10月:}ACM-ICPC亚洲区域赛,银牌
    \item \textbf{2024年5月:}Google Code Jam,全球前1000名
    \item \textbf{2023年11月:}蓝桥杯程序设计大赛,省级一等奖
    \item \textbf{2023年4月:}CCF-CSP认证,满分(300分)
\end{itemize}

\subsection{创新创业比赛}
\begin{itemize}
    \item \textbf{2024年6月:}全国大学生计算机设计大赛,国家级二等奖
    \item \textbf{2024年3月:}校级创新创业大赛,一等奖
    \item \textbf{2023年10月:}"互联网+"大学生创新创业大赛,省级铜奖
\end{itemize}

\subsection{学术奖项}
\begin{itemize}
    \item \textbf{2024年:}国家奖学金获得者
    \item \textbf{2023年:}省级三好学生
    \item \textbf{2023年:}校级优秀学生干部
    \item \textbf{2022-2024年:}连续三年获得校级一等奖学金
\end{itemize}

\section{实习与工作经历}

\subsection{软件开发实习}
\begin{itemize}
    \item \textbf{公司:}腾讯科技(深圳)有限公司
    \item \textbf{时间:}2024年6月 - 2024年8月
    \item \textbf{职位:}后端开发实习生
    \item \textbf{部门:}微信事业群 - 基础架构部
    \item \textbf{工作内容:}
    \begin{itemize}
        \item 参与微信支付核心系统的开发和优化
        \item 负责API网关性能优化,QPS提升20\%
        \item 开发分布式限流组件,应用于多个业务线
        \item 参与系统监控和告警机制的完善
    \end{itemize}
    \item \textbf{技术栈:}Go, Redis, MySQL, Kafka, Prometheus
    \item \textbf{实习评价:}获得优秀实习生称号,收到return offer
\end{itemize}

\subsection{开源社区兼职}
\begin{itemize}
    \item \textbf{组织:}Linux Foundation
    \item \textbf{时间:}2024年1月 - 至今
    \item \textbf{角色:}Community Volunteer
    \item \textbf{工作内容:}
    \begin{itemize}
        \item 帮助维护Kubernetes官方文档
        \item 参与新手导师计划,指导初学者
        \item 翻译技术文档和博客文章
        \item 组织线下技术meetup活动
    \end{itemize}
\end{itemize}

\section{社团活动与领导经历}

\subsection{学生组织}
\begin{itemize}
    \item \textbf{2023年9月 - 2024年6月:}计算机学院学生会技术部部长
    \begin{itemize}
        \item 负责学院官网和信息系统的开发维护
        \item 组织技术培训和编程竞赛
        \item 管理技术部15名成员,协调各项技术支持工作
    \end{itemize}
    \item \textbf{2022年9月 - 2023年6月:}ACM算法协会副会长
    \begin{itemize}
        \item 组织每周算法训练和竞赛辅导
        \item 邀请业界专家进行技术分享
        \item 协会成员在各类竞赛中获奖率提升40\%
    \end{itemize}
\end{itemize}

\subsection{志愿服务}
\begin{itemize}
    \item \textbf{开源软件推广:}参与多个开源项目的推广活动,累计服务时长100+小时
    \item \textbf{技术公益:}为农村中学提供编程教育支持,教授Python基础课程
    \item \textbf{社区服务:}疫情期间开发健康码查验小程序,服务社区居民5000+人
\end{itemize}

\section{技能证书}

\subsection{技术认证}
\begin{itemize}
    \item \textbf{AWS Certified Solutions Architect - Associate}(2024年8月)
    \item \textbf{Oracle Certified Professional Java SE 11 Developer}(2024年3月)
    \item \textbf{Microsoft Azure Fundamentals AZ-900}(2023年12月)
    \item \textbf{Google Cloud Associate Cloud Engineer}(2023年9月)
\end{itemize}

\subsection{语言证书}
\begin{itemize}
    \item \textbf{英语:}CET-6(分数:580)/ 雅思7.0分 / 托福95分
    \item \textbf{日语:}JLPT N2级(计划中)
\end{itemize}

\section{个人特长与兴趣}

\subsection{技术兴趣}
\begin{itemize}
    \item \textbf{开源软件:}关注云原生、容器化、微服务等技术发展
    \item \textbf{算法研究:}热衷于算法优化和数据结构设计
    \item \textbf{人工智能:}学习机器学习和深度学习技术
    \item \textbf{区块链:}研究去中心化应用和智能合约开发
\end{itemize}

\subsection{个人爱好}
\begin{itemize}
    \item \textbf{运动:}篮球、跑步、健身,参加过校内篮球联赛
    \item \textbf{音乐:}吉他演奏,参加过校内音乐节表演
    \item \textbf{摄影:}风景摄影爱好者,作品曾获校级摄影比赛二等奖
    \item \textbf{阅读:}技术书籍、科幻小说、人物传记
\end{itemize}

\section{自我评价}

\subsection{技术能力}
\begin{itemize}
    \item 具有扎实的计算机基础知识和编程能力
    \item 熟悉多种编程语言和开发框架
    \item 有丰富的项目开发和开源贡献经验
    \item 关注技术发展趋势,学习能力强
\end{itemize}

\subsection{个人品质}
\begin{itemize}
    \item \textbf{责任心强:}对承诺的任务能够按时高质量完成
    \item \textbf{团队合作:}具有良好的沟通能力和团队协作精神
    \item \textbf{自驱力强:}主动学习新技术,积极参与开源社区
    \item \textbf{抗压能力:}能够在高压环境下保持高效工作状态
\end{itemize}

\subsection{职业规划}
\begin{itemize}
    \item \textbf{短期目标:}通过OSPP项目深入学习开源技术,提升技术水平
    \item \textbf{中期目标:}成为某个开源项目的核心贡献者,在技术社区有一定影响力
    \item \textbf{长期目标:}在云计算或人工智能领域成为技术专家,推动开源生态发展
\end{itemize}

\section{联系方式}

\begin{table}[H]
\centering
\begin{tabular}{rl}
    \textbf{邮箱:} & \studentemail \\[0.1cm]
    \textbf{电话:} & \studentphone \\[0.1cm]
    \textbf{微信:} & 同手机号 \\[0.1cm]
    \textbf{GitHub:} & \url{https://github.com/yourusername} \\[0.1cm]
    \textbf{Gitee:} & \url{https://gitee.com/yourusername} \\[0.1cm]
    \textbf{LinkedIn:} & \url{https://linkedin.com/in/yourusername} \\[0.1cm]
    \textbf{个人博客:} & \url{https://yourblog.com} \\[0.1cm]
    \textbf{技术公众号:} & YourTechAccount \\
\end{tabular}
\end{table}

\vfill

\begin{center}
    \textcolor{osppgray}{\textit{感谢您花时间阅读我的简历,期待有机会为开源社区贡献力量!}}
\end{center}

\end{document} 