\documentclass[a4paper,12pt]{article}

% ========== 包引入 ==========
% 支持中文
\usepackage[UTF8]{ctex}
\usepackage{geometry}
\usepackage{graphicx}
\usepackage{hyperref}
\usepackage{fancyhdr}
\usepackage{titlesec}
\usepackage{enumitem}
\usepackage{array}              % 表格支持(必须在tabularx之前)
\usepackage{tabularx}
\usepackage{colortbl}
\usepackage{xcolor}
\usepackage{amsmath}
\usepackage{amssymb}
\usepackage{float}
\usepackage{longtable}
\usepackage{booktabs}
\usepackage{multirow}
\usepackage{tikz}
\usepackage{pgfgantt}
\usepackage{fontawesome5}  % 图标支持
\usepackage{tcolorbox}     % 彩色文本框
\usepackage{lipsum}        % 示例文本(可选)

% ========== 页面设置 ==========
\geometry{left=2.5cm,right=2.5cm,top=2.5cm,bottom=2.5cm}

% ========== 基础超链接设置 ==========
\hypersetup{
    colorlinks=true,
    linkcolor=black,        % 目录链接设为黑色
    filecolor=magenta,      
    urlcolor=blue,
    pdftitle={OSPP 2025 项目申请书},
    pdfauthor={您的姓名},
}

% ========== 页眉页脚设置 ==========
\pagestyle{fancy}
\fancyhf{}
\rhead{\small OSPP 2025 项目申请书}
\lhead{\small \studentname}
\cfoot{\thepage}
\renewcommand{\headrulewidth}{0.4pt}

% ========== 标题格式设置 ==========
\titleformat{\section}{\Large\bfseries\color{osppblue}}{\thesection}{1em}{}
\titleformat{\subsection}{\large\bfseries\color{osppdark}}{\thesubsection}{1em}{}
\titleformat{\subsubsection}{\normalsize\bfseries\color{osppgray}}{\thesubsubsection}{1em}{}

% ========== 定义颜色 ==========
\definecolor{osppblue}{RGB}{51, 122, 183}
\definecolor{osppgray}{RGB}{128, 128, 128}

% ========== 自定义命令用于填写个人信息 ==========
\newcommand{\studentname}{您的姓名}
\newcommand{\studentid}{您的学号}
\newcommand{\studentemail}{your.email@example.com}
\newcommand{\studentphone}{您的电话}
\newcommand{\studentuniversity}{您的大学}
\newcommand{\studentmajor}{您的专业}
\newcommand{\projecttitle}{项目标题}
\newcommand{\communityname}{社区名称}
\newcommand{\mentorname}{导师姓名}
\newcommand{\mentoremail}{mentor@example.com}
\newcommand{\projectdifficulty}{基础/进阶/高级}
\newcommand{\projectreward}{奖金金额}

% ========== 引入配置文件 ==========
\IfFileExists{config.tex}{% OSPP 申请书配置文件
% 请在此文件中填写您的个人信息和项目信息

% ========== 个人基本信息 ==========
\renewcommand{\studentname}{张三}                    % 您的姓名
\renewcommand{\studentid}{2021123456}               % 学号
\renewcommand{\studentemail}{zhangsan@example.com}  % 邮箱
\renewcommand{\studentphone}{138-0000-0000}         % 电话
\renewcommand{\studentuniversity}{某某大学}          % 大学
\renewcommand{\studentmajor}{计算机科学与技术}       % 专业

% ========== 项目相关信息 ==========
\renewcommand{\projecttitle}{基于深度学习的开源项目漏洞检测系统}  % 项目标题
\renewcommand{\communityname}{Apache Software Foundation}        % 社区名称
\renewcommand{\mentorname}{李导师}                                % 导师姓名
\renewcommand{\mentoremail}{liteacher@apache.org}               % 导师邮箱
\renewcommand{\projectdifficulty}{进阶}                          % 项目难度:基础/进阶/高级
\renewcommand{\projectreward}{12000元}                          % 项目奖金

% ========== 页面格式设置 ==========
% 修复页眉高度警告
\setlength{\headheight}{14.5pt}

% 页面边距(如需调整请取消注释)
% \geometry{left=3cm,right=3cm,top=3cm,bottom=3cm}

% ========== 颜色设置 ==========
% OSPP 官方主题色(参考官网 https://summer-ospp.ac.cn/)
\definecolor{osppblue}{RGB}{51, 122, 183}      % OSPP主蓝色
\definecolor{osppgray}{RGB}{108, 117, 125}     % OSPP灰色
\definecolor{ospplight}{RGB}{248, 249, 250}    % OSPP浅色背景
\definecolor{osppdark}{RGB}{52, 58, 64}        % OSPP深色文字

% 辅助颜色
\definecolor{successgreen}{RGB}{40, 167, 69}   % 成功绿色
\definecolor{warningorange}{RGB}{255, 193, 7}  % 警告橙色
\definecolor{dangerred}{RGB}{220, 53, 69}      % 危险红色

% ========== 超链接颜色设置 ==========
\hypersetup{
    colorlinks=true,
    linkcolor=black,        % 目录链接改为黑色
    filecolor=osppblue,     
    urlcolor=osppblue,      % 网址链接使用OSPP蓝色
    citecolor=osppblue,     % 引用链接使用OSPP蓝色
    pdftitle={OSPP 2025 项目申请书 - \projecttitle},
    pdfauthor={\studentname},
    pdfsubject={开源软件供应链点亮计划 2025},
    pdfkeywords={OSPP, 开源, 申请书, \studentname}
}

% ========== 字体设置 ==========
% 如果需要使用特定字体,可以在此设置
% \setCJKmainfont{SimSun}                       % 设置中文主字体
% \setCJKsansfont{SimHei}                       % 设置中文无衬线字体

% ========== 自定义命令 ==========
% 定义一些常用的格式命令
\newcommand{\highlight}[1]{\textcolor{osppblue}{\textbf{#1}}}
\newcommand{\important}[1]{\textcolor{dangerred}{\textbf{#1}}}
\newcommand{\note}[1]{\textcolor{osppgray}{\textit{#1}}}
\newcommand{\code}[1]{\texttt{#1}}

% 定义项目相关的命令
\newcommand{\githuburl}{https://github.com/yourusername}
\newcommand{\giteeurl}{https://gitee.com/yourusername}

% ========== 表格样式设置 ==========
% 设置表格的默认样式
\renewcommand{\arraystretch}{1.2}  % 增加表格行间距

% ========== 列表样式设置 ==========
% 自定义列表项符号
\renewcommand{\labelitemi}{\textcolor{osppblue}{\textbullet}}
\renewcommand{\labelitemii}{\textcolor{osppgray}{\textendash}}

% ========== 段落设置 ==========
\setlength{\parindent}{2em}        % 段落缩进
\setlength{\parskip}{0.5em}        % 段落间距 }{}

% ========== TikZ 库设置 ==========
\usetikzlibrary{shapes.geometric, arrows, positioning, calc}

% ========== 自定义样式框 ==========
\tcbuselibrary{most}
\newtcolorbox{tipbox}[1][]{
    colback=ospplight,
    colframe=osppblue,
    title={\faLightbulb \; 提示},
    fonttitle=\bfseries,
    #1
}

\newtcolorbox{warningbox}[1][]{
    colback=yellow!10,
    colframe=orange,
    title={\faExclamationTriangle \; 注意},
    fonttitle=\bfseries,
    #1
}

\newtcolorbox{infobox}[1][]{
    colback=blue!5,
    colframe=osppblue,
    title={\faInfoCircle \; 信息},
    fonttitle=\bfseries,
    #1
}

\begin{document}

% ========== 标题页 ==========
\begin{titlepage}
    \centering
    
    % OSPP Logo 和官方标识
    \vspace{0.5cm}
    
    % OSPP图标(如果存在)
    \IfFileExists{image/OSPP.png}{
        \includegraphics[width=3cm]{image/OSPP.png} \\
        \vspace{0.5cm}
    }{}
    
    % OSPP标题文字
    {\Huge \textcolor{osppblue}{\textbf{开源软件供应链点亮计划}}} \\
    \vspace{0.3cm}
    {\LARGE \textcolor{osppblue}{\textbf{Open Source Promotion Plan}}} \\
    \vspace{0.2cm}
    {\Large \textcolor{osppgray}{\textbf{Summer 2025}}} \\
    \vspace{0.3cm}
    {\small \textcolor{osppgray}{官方网站:\url{https://summer-ospp.ac.cn/}}}
    
    \vspace{2cm}
    
    % 项目标题(去除边框)
    {\huge \textcolor{osppdark}{\textbf{\projecttitle}}}
    
    \vspace{1.5cm}
    
    % 申请信息表格(去除边框)
    \begin{tabular}{rl}
        \textbf{\textcolor{osppblue}{申请学生:}} & \textbf{\studentname} \\[0.3cm]
        \textbf{学号:} & \studentid \\[0.1cm]
        \textbf{邮箱:} & \studentemail \\[0.1cm]
        \textbf{电话:} & \studentphone \\[0.1cm]
        \textbf{大学:} & \studentuniversity \\[0.1cm]
        \textbf{专业:} & \studentmajor \\[0.5cm]
        \textbf{\textcolor{osppblue}{项目信息:}} & \\[0.3cm]
        \textbf{社区:} & \communityname \\[0.1cm]
        \textbf{导师:} & \mentorname \\[0.1cm]
        \textbf{导师邮箱:} & \mentoremail \\[0.1cm]
        \textbf{项目难度:} & \projectdifficulty \\[0.1cm]
        \textbf{项目奖金:} & \projectreward \\
    \end{tabular}
    
    \vfill
    
    {\large \textcolor{osppgray}{申请日期:\today}}
    
\end{titlepage}

% ========== 目录 ==========
\tableofcontents
\newpage

% 正文开始
\section{项目概述}

\subsection{项目背景}
% 请在此处描述项目的背景和意义
在这里详细描述项目的背景,包括:
\begin{itemize}
    \item 项目所在的技术领域
    \item 当前存在的问题或挑战
    \item 项目的重要性和必要性
    \item 预期的影响和价值
\end{itemize}

\subsection{项目目标}
% 请在此处明确说明项目的具体目标
明确列出项目的具体目标:
\begin{enumerate}
    \item 主要目标1
    \item 主要目标2
    \item 主要目标3
\end{enumerate}

\subsection{预期成果}
% 请在此处描述项目完成后的预期成果
描述项目完成后将产生的具体成果,包括但不限于:
\begin{itemize}
    \item 代码贡献
    \item 文档更新
    \item 性能提升
    \item 新功能实现
\end{itemize}

\section{技术方案}

\subsection{现有项目架构分析}

\subsubsection{项目结构分析}
% 根据自己的理解拆解目标开源项目,并能明确描述各个模块的功能
详细分析目标开源项目的结构:

\begin{table}[H]
\centering
\begin{tabular}{|l|l|l|}
\hline
\textbf{模块名称} & \textbf{功能描述} & \textbf{主要文件} \\
\hline
模块A & 处理用户输入和界面交互 & file1.py, file2.py \\
\hline
模块B & 业务逻辑处理和数据验证 & file3.js, file4.js \\
\hline
模块C & 底层数据存储和访问 & file5.cpp, file6.h \\
\hline
\end{tabular}
\caption{项目模块分析表}
\end{table}

\subsubsection{技术栈分析}
分析项目使用的技术栈:
\begin{itemize}
    \item \textbf{编程语言:}Python, JavaScript, C++等
    \item \textbf{框架:}React, Django, Spring等
    \item \textbf{数据库:}MySQL, PostgreSQL, MongoDB等
    \item \textbf{其他工具:}Docker, Kubernetes, Git等
\end{itemize}

\subsection{需求分析}

\subsubsection{功能需求}
% 基于项目拆解,对项目描述中提出的需求进行分析
根据项目描述,详细分析功能需求:

\begin{enumerate}
    \item \textbf{需求1:}详细描述需求1的内容和要求
    \item \textbf{需求2:}详细描述需求2的内容和要求
    \item \textbf{需求3:}详细描述需求3的内容和要求
\end{enumerate}

\subsubsection{非功能需求}
分析性能、安全性、可用性等非功能需求:
\begin{itemize}
    \item \textbf{性能要求:}响应时间、吞吐量等
    \item \textbf{安全要求:}认证、授权、数据保护等
    \item \textbf{可用性要求:}可维护性、可扩展性等
\end{itemize}

\subsection{解决方案设计}

\subsubsection{总体架构设计}
% 明确描述自己的方案将对哪些模块中的哪些文件进行修改
% 或自己的方案将会添加哪些模块,以及这些模块如何与现有模块通信

\textbf{方案概述:}
简要描述解决方案的整体思路和方法。

\textbf{修改的现有模块:}
\begin{table}[H]
\centering
\begin{tabular}{|l|l|l|}
\hline
\textbf{模块} & \textbf{文件} & \textbf{修改内容} \\
\hline
模块A & file1.py & 添加新函数,优化算法 \\
\hline
模块B & file3.js & 修改API接口,参数验证 \\
\hline
\end{tabular}
\caption{现有模块修改计划}
\end{table}

\textbf{新增模块:}
\begin{table}[H]
\centering
\begin{tabular}{|l|l|l|}
\hline
\textbf{新模块} & \textbf{功能描述} & \textbf{通信方式} \\
\hline
新模块D & 数据处理和分析功能 & REST API与模块A通信 \\
\hline
新模块E & 用户界面组件 & 事件总线与模块B通信 \\
\hline
\end{tabular}
\caption{新增模块设计}
\end{table}

\subsubsection{详细设计}

\textbf{算法设计:}
描述关键算法的设计思路和实现方法。

\textbf{数据结构设计:}
说明使用的数据结构和选择理由。

\textbf{接口设计:}
定义模块间的接口规范。

\subsection{技术难点与解决策略}

\subsubsection{预期技术难点}
识别项目实施过程中可能遇到的技术难点:
\begin{enumerate}
    \item \textbf{难点1:}描述技术难点1
    \item \textbf{难点2:}描述技术难点2
    \item \textbf{难点3:}描述技术难点3
\end{enumerate}

\subsubsection{解决策略}
针对每个技术难点提出具体的解决策略:
\begin{itemize}
    \item \textbf{对于难点1:}采用XX技术/方法来解决
    \item \textbf{对于难点2:}参考YY论文/项目的经验
    \item \textbf{对于难点3:}与导师和社区成员讨论
\end{itemize}

\subsection{参考资料}
% Buff加成技巧:以引用的形式在项目申请书中列出其他项目的成功经验、论文等理论依据

\begin{enumerate}
    \item 相关论文:[论文标题] - [作者] - [发表年份]
    \item 开源项目:[项目名称] - [GitHub链接]
    \item 技术文档:[文档标题] - [链接]
    \item 官方文档:[相关技术的官方文档]
\end{enumerate}

\section{时间规划}

\subsection{项目阶段划分}
% 建议时间规划的粒度不大于1周
% 可以先将时间划分为几个大的阶段(如,4周为一个阶段),并总结每个阶段的核心任务

本项目计划分为以下几个主要阶段:

\begin{table}[H]
\centering
\begin{tabular}{|c|c|l|}
\hline
\textbf{阶段} & \textbf{时间周期} & \textbf{核心任务} \\
\hline
第一阶段 & 第1-4周 & 环境搭建、需求分析 \\
\hline
第二阶段 & 第5-8周 & 核心功能开发、单元测试 \\
\hline
第三阶段 & 第9-12周 & 功能完善、集成测试 \\
\hline
第四阶段 & 第13-16周 & 文档编写、项目总结 \\
\hline
\end{tabular}
\caption{项目阶段划分}
\end{table}

\subsection{详细周计划}

\subsubsection{第一阶段:准备与设计(第1-4周)}

\textbf{第1周(日期:XX月XX日 - XX月XX日)}
\begin{itemize}
    \item 熟悉项目代码库,搭建开发环境
    \item 深入研究项目文档和相关技术
    \item 与导师进行首次详细沟通
    \item 完成开发环境配置和工具安装
\end{itemize}

\textbf{第2周(日期:XX月XX日 - XX月XX日)}
\begin{itemize}
    \item 完成需求分析和功能规格说明
    \item 设计详细的技术方案
    \item 制定编码规范和测试策略
    \item 准备第一次进度汇报
\end{itemize}

\textbf{第3周(日期:XX月XX日 - XX月XX日)}
\begin{itemize}
    \item 完成系统架构设计
    \item 设计数据库schema(如需要)
    \item 定义API接口规范
    \item 准备开发所需的第三方库和工具
\end{itemize}

\textbf{第4周(日期:XX月XX日 - XX月XX日)}
\begin{itemize}
    \item 完成详细设计文档
    \item 搭建测试框架
    \item 创建项目开发分支
    \item 第一阶段总结和汇报
\end{itemize}

\subsubsection{第二阶段:核心开发(第5-8周)}

\textbf{第5周(日期:XX月XX日 - XX月XX日)}
\begin{itemize}
    \item 开始核心模块A的开发
    \item 实现基础数据结构和算法
    \item 编写对应的单元测试
    \item 与导师讨论实现细节
\end{itemize}

\textbf{第6周(日期:XX月XX日 - XX月XX日)}
\begin{itemize}
    \item 完成核心模块A的开发
    \item 开始核心模块B的开发
    \item 进行代码审查和重构
    \item 更新相关文档
\end{itemize}

\textbf{第7周(日期:XX月XX日 - XX月XX日)}
\begin{itemize}
    \item 完成核心模块B的开发
    \item 实现模块间的接口和通信
    \item 进行集成测试
    \item 修复发现的bug
\end{itemize}

\textbf{第8周(日期:XX月XX日 - XX月XX日)}
\begin{itemize}
    \item 完成第二阶段的功能开发
    \item 全面的功能测试
    \item 性能基准测试
    \item 第二阶段总结汇报
\end{itemize}

\subsubsection{第三阶段:完善与优化(第9-12周)}

\textbf{第9周(日期:XX月XX日 - XX月XX日)}
\begin{itemize}
    \item 实现剩余的功能模块
    \item 完善错误处理机制
    \item 添加日志和监控功能
    \item 优化用户体验
\end{itemize}

\textbf{第10周(日期:XX月XX日 - XX月XX日)}
\begin{itemize}
    \item 进行全面的系统测试
    \item 性能优化和内存管理
    \item 安全性检查和加固
    \item 兼容性测试
\end{itemize}

\textbf{第11周(日期:XX月XX日 - XX月XX日)}
\begin{itemize}
    \item 代码重构和清理
    \item 完善测试覆盖率
    \item 准备beta版本发布
    \item 收集社区反馈
\end{itemize}

\textbf{第12周(日期:XX月XX日 - XX月XX日)}
\begin{itemize}
    \item 根据反馈进行功能调整
    \item 最终的bug修复
    \item 准备正式版本
    \item 第三阶段总结
\end{itemize}

\subsubsection{第四阶段:收尾与总结(第13-16周)}

\textbf{第13周(日期:XX月XX日 - XX月XX日)}
\begin{itemize}
    \item 完善用户文档和开发者文档
    \item 录制使用演示视频
    \item 准备项目展示材料
    \item 社区推广和宣传
\end{itemize}

\textbf{第14周(日期:XX月XX日 - XX月XX日)}
\begin{itemize}
    \item 最终代码审查和优化
    \item 部署到生产环境(如适用)
    \item 编写技术博客和分享文章
    \item 准备项目总结报告
\end{itemize}

\textbf{第15周(日期:XX月XX日 - XX月XX日)}
\begin{itemize}
    \item 项目成果展示和汇报
    \item 与社区成员分享经验
    \item 讨论后续维护计划
    \item 收集项目反馈和建议
\end{itemize}

\textbf{第16周(日期:XX月XX日 - XX月XX日)}
\begin{itemize}
    \item 完成最终项目报告
    \item 整理项目资料和代码
    \item 参与OSPP结项答辩
    \item 项目正式结束和交接
\end{itemize}

\subsection{风险评估与应对措施}
% 可以预留15%弹性时间,给debug留足空间

\textbf{时间风险:}
\begin{itemize}
    \item \textbf{风险:}某些功能实现比预期复杂
    \item \textbf{应对:}在每个阶段预留15\%的弹性时间用于debug和调整
\end{itemize}

\textbf{技术风险:}
\begin{itemize}
    \item \textbf{风险:}遇到难以解决的技术问题
    \item \textbf{应对:}及时与导师沟通,寻求社区帮助,准备备选方案
\end{itemize}

\textbf{依赖风险:}
\begin{itemize}
    \item \textbf{风险:}第三方库或工具出现问题
    \item \textbf{应对:}提前调研备选方案,保持灵活性
\end{itemize}

\section{附录}

\subsection{项目时间表(甘特图)}
% 使用pgfgantt包创建甘特图
\begin{figure}[H]
\centering
\begin{ganttchart}[
    x unit=0.8cm,
    y unit title=0.5cm,
    y unit chart=0.4cm,
    hgrid,
    vgrid,
    title height=1,
    bar height=0.3,
    group height=0.3
]{1}{16}

% 标题
\gantttitle{OSPP 2025 项目时间规划}{16} \\
\gantttitle{第1-4周}{4}
\gantttitle{第5-8周}{4}
\gantttitle{第9-12周}{4}
\gantttitle{第13-16周}{4} \\

% 任务组
\ganttgroup{第一阶段:准备与设计}{1}{4} \\
\ganttbar{环境搭建}{1}{1} \\
\ganttbar{需求分析}{2}{2} \\
\ganttbar{架构设计}{3}{3} \\
\ganttbar{详细设计}{4}{4} \\

\ganttgroup{第二阶段:核心开发}{5}{8} \\
\ganttbar{模块A开发}{5}{6} \\
\ganttbar{模块B开发}{6}{7} \\
\ganttbar{集成测试}{7}{8} \\

\ganttgroup{第三阶段:完善与优化}{9}{12} \\
\ganttbar{功能完善}{9}{10} \\
\ganttbar{性能优化}{10}{11} \\
\ganttbar{系统测试}{11}{12} \\

\ganttgroup{第四阶段:收尾与总结}{13}{16} \\
\ganttbar{文档编写}{13}{14} \\
\ganttbar{项目总结}{14}{15} \\
\ganttbar{成果展示}{15}{16} \\

% 里程碑
\ganttmilestone{设计完成}{4}
\ganttmilestone{核心功能完成}{8}
\ganttmilestone{系统完成}{12}
\ganttmilestone{项目结束}{16}

\end{ganttchart}
\caption{项目时间规划甘特图}
\end{figure}

\subsection{技术架构图}
% 这里可以插入技术架构图
\begin{figure}[H]
\centering
% 如果有架构图,可以用以下方式插入
% \includegraphics[width=0.8\textwidth]{architecture.png}
\begin{tikzpicture}[node distance=2cm, auto]
    % 定义节点样式
    \tikzstyle{box} = [rectangle, draw, fill=blue!20, text width=3cm, text centered, rounded corners, minimum height=1cm]
    \tikzstyle{arrow} = [thick,->,>=stealth]
    
    % 创建节点
    \node [box] (frontend) {前端界面};
    \node [box, below of=frontend] (api) {API层};
    \node [box, below of=api] (service) {业务逻辑层};
    \node [box, below of=service] (data) {数据访问层};
    \node [box, below of=data] (database) {数据库};
    
    % 创建箭头
    \draw [arrow] (frontend) -- (api);
    \draw [arrow] (api) -- (service);
    \draw [arrow] (service) -- (data);
    \draw [arrow] (data) -- (database);
\end{tikzpicture}
\caption{系统架构图示例}
\end{figure}

\subsection{联系方式}
如有任何问题,请通过以下方式联系:
\begin{itemize}
    \item \textbf{邮箱:}\studentemail
    \item \textbf{电话:}\studentphone
    \item \textbf{GitHub:}https://github.com/yourusername
    \item \textbf{社区交流群:}XXXX(微信/QQ群号)
\end{itemize}

\section{申请动机}

\subsection{选择该项目的原因}
详细说明为什么选择这个特定的项目:
\begin{itemize}
    \item 与个人技术兴趣和专业方向的契合度
    \item 对该技术领域的热情和理解
    \item 希望通过项目学习和掌握的技能
    \item 对开源社区的认识和参与意愿
\end{itemize}

\subsection{期望与目标}
\begin{itemize}
    \item \textbf{技术目标:}希望通过项目掌握XX技术,提升XX能力
    \item \textbf{个人发展:}希望在开源贡献、团队协作等方面得到锻炼
    \item \textbf{社区贡献:}希望为开源社区做出有意义的贡献
    \item \textbf{长期规划:}项目结束后继续参与社区,成为长期贡献者
\end{itemize}

\subsection{承诺与保证}
\begin{itemize}
    \item 保证每周投入至少30小时进行项目开发
    \item 按时完成各阶段的任务和交付物
    \item 积极与导师沟通,主动寻求帮助和指导
    \item 遵守开源社区的行为准则和开发规范
    \item 项目结束后继续维护和改进代码
\end{itemize}

\section{个人简历}

\textcolor{osppblue}{\textbf{注:}}个人简历已独立为一份文档,请参阅《\textcolor{osppblue}{\textbf{个人简历.pdf}}》文件。

该简历包含以下内容:
\begin{itemize}
    \item 基本信息和教育背景
    \item 技术技能和开发经验
    \item 开源贡献和项目经历
    \item 竞赛获奖和实习经历
    \item 社团活动和志愿服务
    \item 技能证书和自我评价
\end{itemize}

简历文件可通过以下方式编译生成:
\begin{itemize}
    \item 使用 \texttt{pdflatex personal-resume.tex} 命令
    \item 或使用 \texttt{xelatex personal-resume.tex} 命令(推荐)
\end{itemize}

\end{document} 